%%%%%%%%%%%%%%%%%%%%%%%%%%%%%%%%%%%%%%%%%%%%%%%%%%%%%%%%%%%%%%%%%%%%%%%%%%%%%%%
%                                                                             %
% Headers for the document, it defines the main style and some macros to make %
% easy some things.                                                           %
%                                                                             %
% Cristian Recabarren 2016                                                    %
%                                                                             %
%%%%%%%%%%%%%%%%%%%%%%%%%%%%%%%%%%%%%%%%%%%%%%%%%%%%%%%%%%%%%%%%%%%%%%%%%%%%%%%

\usepackage[utf8]{inputenc}
\usepackage[english,greek,spanish]{babel}
\usepackage[pdftex]{graphicx}
\usepackage{float}
\usepackage{setspace}
\ifdefined\FINAL{}
\usepackage[paperwidth=216mm,paperheight=330mm,tmargin=25mm,bmargin=30mm,lmargin=30mm,rmargin=20mm,footskip=10mm,headsep=0mm]{geometry}
\else
\usepackage[paperwidth=216mm,paperheight=279mm,tmargin=25mm,bmargin=30mm,lmargin=30mm,rmargin=20mm,footskip=10mm,headsep=0mm]{geometry}
\fi 
\usepackage{tocbibind}
%\usepackage{cite}
\usepackage[square,super,sort]{natbib}
\usepackage{datetime}
\usepackage{fancyref}
\usepackage{nameref}
\usepackage{amsmath}
\usepackage{amsthm}
\usepackage{amsfonts}
\usepackage[all]{xy}
\usepackage{caption}
\usepackage{subcaption}
\usepackage[xspace]{ellipsis}
\usepackage[table, usenames, dvipsnames]{xcolor}
\usepackage{array,ragged2e}
\usepackage[xindy,toc,acronym]{glossaries}
\usepackage[xindy]{imakeidx}
\usepackage{multirow}
\usepackage{longtable}

%-------------------------------------------------------------------------------

\definecolor{lightgray}{gray}{0.82}
\definecolor{gray}{gray}{0.72}

%-------------------------------------------------------------------------------

\makeglossaries{}
\makeindex{}

%-------------------------------------------------------------------------
%definición de comandos

\newcommand{\dotrule}[1]{\parbox[t]{#1}{\dotfill}}

%para código fuente
\newenvironment{codigoenv}
{\fontsize{10pt}{12pt} \linespread{1}} { \normalsize}

%para las imagenes
\newenvironment{figuraenv}
{\begin{figure}[h!]\begin{center}} {\end{center}\end{figure}}

\graphicspath{{./images/}}

%para las ecuaciones
%\newaliascnt{eqfloat}{equation}
%\newfloat{eqfloat}{h}{eqflts}
%\floatname{eqfloat}{Equation}
%\newcommand*{\ORGeqfloat}{}
%\let\ORGeqfloat\eqfloat
%\def\eqfloat{%
%  \let\ORIGINALcaption\caption
%  \def\caption{%
%    \addtocounter{equation}{-1}%
%    \ORIGINALcaption
%  }%
%  \ORGeqfloat
%}

\newcommand{\df}[2]{\textit{#1 (#2)}} %definicion de termino y sigla
\newcommand{\cls}[1]{\mbox{\textit{#1}}} %nombre de clase o paquete o código
\newcommand{\trm}[1]{\textit{#1}} %termino tecnico o en ingles
\newcommand{\cod}[1]{\texttt{\footnotesize #1}} %codigo fuente

%-------------------------------------------------------------------------
%\newcommand{\textgreek}[1]
%    {\bgroup\greekfont{#1}\egroup} % Greek text
%-------------------------------------------------------------------------

\linespread{1}
\setlength{\parskip}{1\baselineskip}
\setlength{\parindent}{1cm}
\sloppy

%%%%%%%%%%%%%%%%%%%%%%%%%%%%%%
\newenvironment{dedication}
{
   \cleardoublepage{}
   \thispagestyle{empty}
   \vspace*{\stretch{1}}
   \hfill\begin{minipage}[t]{0.66\textwidth}
   \raggedright{}
}%
{
   \end{minipage}
   \vspace*{\stretch{3}}
   \clearpage
}
