%%%%%%%%%%%%%%%%%%%%%%%%%%%%%%%%%%%%%%%%%%%%%%%%%%%%%%%%%%%%%%%%%%%%%%%%%%%%%%%
%                                                                             %
% Headers for the document, it defines the main style and some macros to make %
% easy some things.                                                           %
%                                                                             %
% Cristian Recabarren 2016                                                    %
%                                                                             %
%%%%%%%%%%%%%%%%%%%%%%%%%%%%%%%%%%%%%%%%%%%%%%%%%%%%%%%%%%%%%%%%%%%%%%%%%%%%%%%
\usepackage[utf8]{inputenc}
\usepackage[english,greek,spanish,es-tabla]{babel}
\usepackage[pdftex]{graphicx}
\usepackage{hyperref,xpatch} %[unicode=true]
\usepackage[titletoc,toc,page]{appendix}
\usepackage{float}
\usepackage{setspace}
\usepackage[paperwidth=216mm,paperheight=330mm,tmargin=40mm,bmargin=40mm,lmargin=50mm,rmargin=20mm,footskip=10mm,headsep=0mm]{geometry}
\usepackage{tocbibind}
\usepackage[square,super,sort]{natbib}
\usepackage{datetime}
\usepackage{fancyref}
\usepackage{amsmath}
\usepackage{amsthm}
\usepackage{amsfonts}
\usepackage[all]{xy}
\usepackage{caption}
\usepackage{subcaption}
\usepackage[xspace]{ellipsis}
\usepackage[table, usenames, dvipsnames]{xcolor}
\usepackage{array,ragged2e}
\usepackage[xindy,toc,acronym]{glossaries}
\usepackage[xindy]{imakeidx}
\usepackage{multirow}
\usepackage{longtable}
\usepackage{wrapfig}
\usepackage{tikz}
\usepackage{rotating}
\usepackage{textcomp}
\usepackage{xcolor}
\usepackage{colortbl}
%-------------------------------------------------------------------------------
\def\checkmark{\tikz\fill[scale=0.4](0,.35) -- (.25,0) -- (1,.7) -- (.25,.15) -- cycle;}
%-------------------------------------------------------------------------------
\definecolor{lightgray}{gray}{0.82}
\definecolor{gray}{gray}{0.72}
%-------------------------------------------------------------------------------
\makeglossaries{}
\makeindex{}
%-------------------------------------------------------------------------
%definición de comandos
\newcommand{\dotrule}[1]{\parbox[t]{#1}{\dotfill}}
%para código fuente
\newenvironment{codigoenv}
{\fontsize{10pt}{12pt} \linespread{1}} { \normalsize}
%para las imágenes
\newenvironment{figuraenv}
{\begin{figure}[hbt!]\begin{center}} {\end{center}\end{figure}}
%ubicación de las imágenes
\graphicspath{{./images/}}
%---------------------------------------------------------------------------
\newcommand{\df}[2]{\textit{#1 (#2)}} %definicion de termino y sigla
\newcommand{\cls}[1]{\mbox{\textit{#1}}} %nombre de clase o paquete o código
\newcommand{\trm}[1]{\textit{#1}} %termino tecnico o en ingles
\newcommand{\cod}[1]{\texttt{\footnotesize #1}} %codigo fuente

\linespread{1}
\setlength{\parskip}{1\baselineskip}
\setlength{\parindent}{1cm}
\sloppy

%%%%%%%%%%%%%%%%%%%%%%%%%%%%%%
\newenvironment{dedication}
{
   \cleardoublepage{}
   \thispagestyle{empty}
   \vspace*{\stretch{1}}
   \hfill\begin{minipage}[t]{0.66\textwidth}
   \raggedright{}
}%
{
   \end{minipage}
   \vspace*{\stretch{3}}
   \clearpage
}
