%%%%%%%%%%%%%%%%%%%%%%%%%%%%%%%%%%%%%%%%%%%%%%%%%%%%%%%%%%%%%%%%%%%%%%%%%%%%%%%
%                                                                             %
%                                                                             %
%                                                                             %
% Cristian Recabarren 2016                                                    %
%                                                                             %
%%%%%%%%%%%%%%%%%%%%%%%%%%%%%%%%%%%%%%%%%%%%%%%%%%%%%%%%%%%%%%%%%%%%%%%%%%%%%%%

%-------------------------------------------------------------------------
\chapter{Marco Teórico y Metodología}
\label{chap:MarcoTeorico}
Los textos en \index{\LaTeX}\LaTeX~se componen de diversas partes, pero básicamente podemos dividirlas en dos: el \index{preámbulo}preámbulo y el documento en sí. En el preámbulo se definen comandos y los paquetes de latex que se utilizarán para escribir el documento. Este documento define algunos comandos y directorios para facilitar algunas cosas.

\section{Introducción}
\label{sec:introduction}

\begin{figuraenv}
\includegraphics[width=10cm]{llama.jpg}
\caption[This label goes to the TOC]{This description goes bellow the inserted image}\label{label to reference the image into text}
\end{figuraenv}

\begin{longtable}{| r | p{4cm} | p{4cm} |}
  \hline
  \cellcolor{lightgray}\textbf{} & \cellcolor{lightgray}\textbf{Forma Diferencial} & \cellcolor{lightgray}\textbf{Forma Integral} \\
  \hline
  \endhead
  \cellcolor{lightgray}\textbf{1.ª Ley de Maxwell} & 
  \begin{equation}
    \vec{\nabla}\cdot\vec{E} = 0
  \end{equation} &
  \begin{equation}
    \oint_{S} \vec{E} \cdot d\vec{S} = 0
  \end{equation} \\
  \hline
  \cellcolor{lightgray}\textbf{2.ª Ley de Maxwell} & 
  \begin{equation}
    \vec{\nabla}\cdot\vec{B} = 0
  \end{equation} &
  \begin{equation}
    \oint_{S} \vec{B} \cdot d\vec{S} = 0
  \end{equation} \\
  \hline
  \cellcolor{lightgray}\textbf{3.ª Ley de Maxwell} & 
  \begin{equation}
    \vec{\nabla}\times\vec{E} = -\dfrac{\partial \vec{B}}{\partial t}
  \end{equation} &
  \begin{equation}
    \oint_{C} \vec{E} \cdot d\vec{l} = - \dfrac{d\Phi_{B}}{dt}
  \end{equation} \\
  \hline
  \cellcolor{lightgray}\textbf{4.ª Ley de Maxwell} & 
  \begin{equation}
    \vec{\nabla}\times\vec{B} = \mu\varepsilon \dfrac{\partial \vec{E}}{\partial t}
  \end{equation} &
  \begin{equation}
    \oint_{C} \vec{B} \cdot d\vec{l} = \mu\varepsilon \dfrac{d\Phi_{E}}{dt}
  \end{equation} \\
  \hline
  \caption[Leyes de Maxwell en ausencia de cargas y corrientes eleéctricas]{Leyes de Maxwell en ausencia de cargas y corrientes eléctricas}\label{tab:MaxwellsLaws}
\end{longtable}
